
\documentclass[11pt,a4paper]{report}

\usepackage[portuges]{babel}
\usepackage[utf8]{inputenc} % define o encoding usado texto fonte (input)--usual "utf8" ou "latin1
\usepackage{graphicx} %permite incluir graficos, tabelas, figuras
\usepackage{subcaption}
\usepackage{listings}
\usepackage{color}
\usepackage{multicol}
\usepackage{indentfirst}

\definecolor{myblue}{rgb}{0.2,0.2,0.8}
\definecolor{mygray}{rgb}{0.5,0.5,0.5}
\definecolor{mymauve}{rgb}{0.58,0,0.82}

\lstdefinestyle{code}{ 
  backgroundcolor=\color{white},   % choose the background color; you must add \usepackage{color} or \usepackage{xcolor}; should come as last argument
  basicstyle=\footnotesize,        % the size of the fonts that are used for the code
  breakatwhitespace=false,         % sets if automatic breaks should only happen at whitespace
  breaklines=true,                 % sets automatic line breaking
  captionpos=b,                    % sets the caption-position to bottom
  commentstyle=\color{white},      % comment style
  deletekeywords={...},            % if you want to delete keywords from the given language
  escapeinside={\%*}{*)},          % if you want to add LaTeX within your code
  extendedchars=true,              % lets you use non-ASCII characters; for 8-bits encodings only, does not work with UTF-8
  firstnumber=1000,                % start line enumeration with line 1
  keepspaces=true,                 % keeps spaces in text, useful for keeping indentation of code (possibly needs columns=flexible)
  keywordstyle=\color{blue},       % keyword style
  language=C++,                    % the language of the code
  morekeywords={*,...},            % if you want to add more keywords to the set
  numberstyle=\tiny\color{mygray}, % the style that is used for the line-numbers
  rulecolor=\color{black},         % if not set, the frame-color may be changed on line-breaks within not-black text (e.g. comments (green here))
  showspaces=false,                % show spaces everywhere adding particular underscores; it overrides 'showstringspaces'
  showstringspaces=false,          % underline spaces within strings only
  showtabs=false,                  % show tabs within strings adding particular underscores
  stepnumber=2,                    % the step between two line-numbers. If it's 1, each line will be numbered
  stringstyle=\color{mymauve},     % string literal style
  tabsize=2,	                   % sets default tabsize to 2 spaces
  title=\lstname                   % show the filename of files included with \lstinputlisting; also try caption instead of title
}

\lstdefinestyle{xml}{ 
  backgroundcolor=\color{white},   % choose the background color; you must add \usepackage{color} or \usepackage{xcolor}; should come as last argument
  basicstyle=\footnotesize,        % the size of the fonts that are used for the code
  breakatwhitespace=false,         % sets if automatic breaks should only happen at whitespace
  breaklines=true,                 % sets automatic line breaking
  captionpos=b,                    % sets the caption-position to bottom
  commentstyle=\color{white},    % comment style
  deletekeywords={...},            % if you want to delete keywords from the given language
  escapeinside={\%*}{*)},          % if you want to add LaTeX within your code
  extendedchars=true,              % lets you use non-ASCII characters; for 8-bits encodings only, does not work with UTF-8
  firstnumber=1000,                % start line enumeration with line 1
  keepspaces=true,                 % keeps spaces in text, useful for keeping indentation of code (possibly needs columns=flexible)
  keywordstyle=\color{blue},       % keyword style
  language=XML,                 % the language of the code
  morekeywords={*,...},            % if you want to add more keywords to the set
  numberstyle=\tiny\color{mygray}, % the style that is used for the line-numbers
  rulecolor=\color{black},         % if not set, the frame-color may be changed on line-breaks within not-black text (e.g. comments (green here))
  showspaces=false,                % show spaces everywhere adding particular underscores; it overrides 'showstringspaces'
  showstringspaces=false,          % underline spaces within strings only
  showtabs=false,                  % show tabs within strings adding particular underscores
  stepnumber=2,                    % the step between two line-numbers. If it's 1, each line will be numbered
  stringstyle=\color{mymauve},     % string literal style
  tabsize=2,	                   % sets default tabsize to 2 spaces
  title=\lstname                   % show the filename of files included with \lstinputlisting; also try caption instead of title
}

\title{Computação Gráfica (3º ano de Curso)\\
       \textbf{Fase 3}\\ Relatório de Desenvolvimento
       } %Titulo do documento
%\title{Um Exemplo de Artigo em \LaTeX}
\author{Diogo Fernandes\\ (A87968) \and Luís Guimarães\\ (A87947)
         \and Ivo Lima\\ (A90214)
       } %autores do documento
\date{\today} %data

\begin{document}
	\begin{minipage}{0.9\linewidth}
        \centering
		\includegraphics[width=0.4\textwidth]{um.jpeg}\par\vspace{1cm}
		{\scshape\LARGE Universidade do Minho} \par
		\vspace{0.6cm}
		{\scshape\Large Licenciatura em Ciências da Computação} \par
		\maketitle
	\end{minipage}


\tableofcontents % insere Indice

\chapter{Introdução}

Nesta terceira fase finalzamos mais uma parte da \emph{engine}, medida em que possibilitamos aos planetas do nosso Sistema Solar realizarem a sua translação à volta do Sol e ainda incluimos, faltando-nos portanto a última fase do desenvolvimento que será a 4 fase que pedirá o acrescento de texturas e luzes à \emph{scene}. 
Como tal, podemos estabelecer um conjunto de tarefas que seram aplicadas tanto no \emph{generator} como na \emph{engine} para suportarem os novos requisitos indicados no enunciado. 
Atualizações à \emph{generator}:
Atualizações à \emph{engine}:
Expandir o processamento do ficheiro \emph{xml} de modo a suportar um novo tipo de \emph{translates} e \emph{rotates}.
Aplicar o conhecimento adquirido sobre curvas de emph{catmull-rom} na expansão da primitiva \emph{translate} para que os modelos se movimentem, tendo por base um conjunto de pontos e o tempo.

\chapter{Atualização do \emph{generator}}
\section{Nova \emph{feature} de criação de um pacth de Bezier}


\chapter{Atualização da \emph{engine}}
\section{Nova configuração do xml}
O ficheiro de configuração do \emph{xml} que temos vindo a utilizar dá suporte a um novo tipo de translações, utilizando curvas (interpolação) de \emph{catmull-rom} , e rotações de modelos para a simulação das órbitas dos planetas e a rotação dos mesmo sob o seu próprio eixo durante um certo período de tempo.
Temos portanto as seguintes atualizações:

• Translações com base em curvas cúbicas de \emph{catmull-rom}, fornecendo:

1. O tempo total para percorrer a curva toda;

2. Um conjunto de pontos de controlo de P0 a Pn com n $\ge$ 4;

Partindo da estrutura seguinte:

\begin{lstlisting}[style = xml]
<translate time = "...">
      <point X = "..." Y = "..." Z = "..."/>
      <point X = "..." Y = "..." Z = "..."/>
      ...
      <point X = "..." Y = "..." Z = "..."/>
      <point X = "..." Y = "..." Z = "..."/>
</translate>
\end{lstlisting}

• Rotações com base no tempo de rotação (360 graus) segundo um (ou mais) eixo(s):

1. O tempo total, da mesma forma que nas translações;

2. O(s) eixo(s) para orientar a rotação do modelo.

\begin{lstlisting}[style = xml]
<rotate time="..." X="..." Y="..." Z="..."/>
\end{lstlisting}
\chapter{Sistema Solar e órbitas dos planetas}

\chapter{Conclusão}

Graças a esta terceira fase o nosso projeto ganhou uma maior capacidade de processamento tendo para tal utilizado uma série de conceitos teóricos, que passaram pelas  curvas e superfícies de \emph{Bezier} até às interpolações de \emph{Catmull-Rom}.

De facto, com esta fase temos uma nova visão do Sistema Solar, pois o mesmo possui de momento um dinamismo concedido pela rotação e translação dos planetas, sendoque para tal foi necessário calcular os pontos intermédios a partir de valores de controlo, atualizar a configuração do \emph{xml} e otimizar o \emph{engine} com o buffering dos vértices no GPU usando VBOs.

A leitura dos ficheiros patch e a estraturação de uma estratégia para os VBOs com índices foi importante para a consolidação das matérias apresentadas nos guiões práticos.

Em suma, esta fase revelou-se fulcral para o desenvolvimento do projeto, embora esta não tenha sido a fase final, este sistema serve como uma boa base para a última etapa, que pedirá texturas e luzes. Consideramos que os objetivos definidos para a terceira fase foram cumpridos na sua íntegra.

\end{document}
