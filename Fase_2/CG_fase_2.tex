
\documentclass[11pt,a4paper]{report}

\usepackage[portuges]{babel}
\usepackage[utf8]{inputenc} % define o encoding usado texto fonte (input)--usual "utf8" ou "latin1
\usepackage{graphicx} %permite incluir graficos, tabelas, figuras
\usepackage{subcaption}
\usepackage{listings}
\usepackage{color}
\usepackage{multicol}

\definecolor{myblue}{rgb}{0.2,0.2,0.8}
\definecolor{mygray}{rgb}{0.5,0.5,0.5}
\definecolor{mymauve}{rgb}{0.58,0,0.82}

\lstdefinestyle{code}{ 
  backgroundcolor=\color{white},   % choose the background color; you must add \usepackage{color} or \usepackage{xcolor}; should come as last argument
  basicstyle=\footnotesize,        % the size of the fonts that are used for the code
  breakatwhitespace=false,         % sets if automatic breaks should only happen at whitespace
  breaklines=true,                 % sets automatic line breaking
  captionpos=b,                    % sets the caption-position to bottom
  commentstyle=\color{white},    % comment style
  deletekeywords={...},            % if you want to delete keywords from the given language
  escapeinside={\%*}{*)},          % if you want to add LaTeX within your code
  extendedchars=true,              % lets you use non-ASCII characters; for 8-bits encodings only, does not work with UTF-8
  firstnumber=1000,                % start line enumeration with line 1
  keepspaces=true,                 % keeps spaces in text, useful for keeping indentation of code (possibly needs columns=flexible)
  keywordstyle=\color{blue},       % keyword style
  language=C++,                 % the language of the code
  morekeywords={*,...},            % if you want to add more keywords to the set
  numberstyle=\tiny\color{mygray}, % the style that is used for the line-numbers
  rulecolor=\color{black},         % if not set, the frame-color may be changed on line-breaks within not-black text (e.g. comments (green here))
  showspaces=false,                % show spaces everywhere adding particular underscores; it overrides 'showstringspaces'
  showstringspaces=false,          % underline spaces within strings only
  showtabs=false,                  % show tabs within strings adding particular underscores
  stepnumber=2,                    % the step between two line-numbers. If it's 1, each line will be numbered
  stringstyle=\color{mymauve},     % string literal style
  tabsize=2,	                   % sets default tabsize to 2 spaces
  title=\lstname                   % show the filename of files included with \lstinputlisting; also try caption instead of title
}

\lstdefinestyle{calculations}{
  breaklines=true,
  basicstyle=\footnotesize,
  extendedchars=true,
  keepspaces=true  
}

\title{Computação Gráfica (3º ano de Curso)\\
       \textbf{Fase 2}\\ Relatório de Desenvolvimento
       } %Titulo do documento
%\title{Um Exemplo de Artigo em \LaTeX}
\author{Diogo Fernandes\\ (A87968) \and Luís Guimarães\\ (A87947)
         \and Ivo Lima\\ (A90214)
       } %autores do documento
\date{\today} %data

\begin{document}
	\begin{minipage}{0.9\linewidth}
        \centering
		\includegraphics[width=0.4\textwidth]{um.jpeg}\par\vspace{1cm}
		{\scshape\LARGE Universidade do Minho} \par
		\vspace{0.6cm}
		{\scshape\Large Licenciatura em Ciências da Computação} \par
		\maketitle
	\end{minipage}

\tableofcontents % insere Indice

\chapter{Introdução}

Nesta segunda fase tivemos como principal objetivo a criação de um cenário hierárquico, onde uma certa cena (no nosso caso um modelo do Sistema Solar) será definido a partir de uma árvore onde cada nodo irá conter um conjunto de transformações geométricas (translações,rotações e escalas). Cada nodo terá vários nodos filhos.

A única parte que sofreu alterações foi o \emph{xml} para que fosse possível processar os novos elementos e torna-se capaz de ler um ficheiro de configuração na \emph{engine}.

\chapter{Conclusão}

Graças a esta segunda fase o nosso projeto ganhou uma maior capacidade de processamento através dos ficheiros de configuração do \emph{xml} .
As transformações pedidas (translações,rotações e escalas) já são possíveis,  o que atribuiu ao nossa modelo de Sistema Solar um atributo visual mais rico e concedeu-lhe vida.

Concluímos assim que esta fase foi fulcral para a nossa maquete e que embora este estágio ainda seja muito prematura este sistema será de grande importância para as próximas fases, logo o nosso propósito nesta etapa foi completado.

\end{document}