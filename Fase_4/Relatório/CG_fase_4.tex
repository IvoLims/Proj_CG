
\documentclass[11pt,a4paper]{report}

\usepackage[portuges]{babel}
\usepackage[utf8]{inputenc} % define o encoding usado texto fonte (input)--usual "utf8" ou "latin1
\usepackage{graphicx} %permite incluir graficos, tabelas, figuras
\usepackage{subcaption}
\usepackage{listings}
\usepackage{color}
\usepackage{multicol}
\usepackage{indentfirst}
\usepackage{multicol}

\definecolor{myblue}{rgb}{0.2,0.2,0.8}
\definecolor{mygray}{rgb}{0.5,0.5,0.5}
\definecolor{mymauve}{rgb}{0.58,0,0.82}

\lstdefinestyle{code}{ 
  backgroundcolor=\color{white},   % choose the background color; you must add \usepackage{color} or \usepackage{xcolor}; should come as last argument
  basicstyle=\footnotesize,        % the size of the fonts that are used for the code
  breakatwhitespace=false,         % sets if automatic breaks should only happen at whitespace
  breaklines=true,                 % sets automatic line breaking
  captionpos=b,                    % sets the caption-position to bottom
  commentstyle=\color{white},      % comment style
  deletekeywords={...},            % if you want to delete keywords from the given language
  escapeinside={\%*}{*)},          % if you want to add LaTeX within your code
  extendedchars=true,              % lets you use non-ASCII characters; for 8-bits encodings only, does not work with UTF-8
  firstnumber=1000,                % start line enumeration with line 1
  keepspaces=true,                 % keeps spaces in text, useful for keeping indentation of code (possibly needs columns=flexible)
  keywordstyle=\color{blue},       % keyword style
  language=C++,                    % the language of the code
  morekeywords={*,...},            % if you want to add more keywords to the set
  numberstyle=\tiny\color{mygray}, % the style that is used for the line-numbers
  rulecolor=\color{black},         % if not set, the frame-color may be changed on line-breaks within not-black text (e.g. comments (green here))
  showspaces=false,                % show spaces everywhere adding particular underscores; it overrides 'showstringspaces'
  showstringspaces=false,          % underline spaces within strings only
  showtabs=false,                  % show tabs within strings adding particular underscores
  stepnumber=2,                    % the step between two line-numbers. If it's 1, each line will be numbered
  stringstyle=\color{mymauve},     % string literal style
  tabsize=2,	                   % sets default tabsize to 2 spaces
  title=\lstname                   % show the filename of files included with \lstinputlisting; also try caption instead of title
}

\lstdefinestyle{xml}{ 
  backgroundcolor=\color{white},   % choose the background color; you must add \usepackage{color} or \usepackage{xcolor}; should come as last argument
  basicstyle=\footnotesize,        % the size of the fonts that are used for the code
  breakatwhitespace=false,         % sets if automatic breaks should only happen at whitespace
  breaklines=true,                 % sets automatic line breaking
  captionpos=b,                    % sets the caption-position to bottom
  commentstyle=\color{white},    % comment style
  deletekeywords={...},            % if you want to delete keywords from the given language
  escapeinside={\%*}{*)},          % if you want to add LaTeX within your code
  extendedchars=true,              % lets you use non-ASCII characters; for 8-bits encodings only, does not work with UTF-8
  firstnumber=1000,                % start line enumeration with line 1
  keepspaces=true,                 % keeps spaces in text, useful for keeping indentation of code (possibly needs columns=flexible)
  keywordstyle=\color{blue},       % keyword style
  language=XML,                 % the language of the code
  morekeywords={*,...},            % if you want to add more keywords to the set
  numberstyle=\tiny\color{mygray}, % the style that is used for the line-numbers
  rulecolor=\color{black},         % if not set, the frame-color may be changed on line-breaks within not-black text (e.g. comments (green here))
  showspaces=false,                % show spaces everywhere adding particular underscores; it overrides 'showstringspaces'
  showstringspaces=false,          % underline spaces within strings only
  showtabs=false,                  % show tabs within strings adding particular underscores
  stepnumber=2,                    % the step between two line-numbers. If it's 1, each line will be numbered
  stringstyle=\color{mymauve},     % string literal style
  tabsize=2,	                   % sets default tabsize to 2 spaces
  title=\lstname                   % show the filename of files included with \lstinputlisting; also try caption instead of title
}

\lstdefinestyle{cmake}{ 
  backgroundcolor=\color{white},   % choose the background color; you must add \usepackage{color} or \usepackage{xcolor}; should come as last argument
  basicstyle=\footnotesize,        % the size of the fonts that are used for the code
  breakatwhitespace=false,         % sets if automatic breaks should only happen at whitespace
  breaklines=true,                 % sets automatic line breaking
  captionpos=b,                    % sets the caption-position to bottom
  commentstyle=\color{white},      % comment style
  deletekeywords={...},            % if you want to delete keywords from the given language
  escapeinside={\%*}{*)},          % if you want to add LaTeX within your code
  extendedchars=true,              % lets you use non-ASCII characters; for 8-bits encodings only, does not work with UTF-8
  firstnumber=1000,                % start line enumeration with line 1
  keepspaces=true,                 % keeps spaces in text, useful for keeping indentation of code (possibly needs columns=flexible)
  keywordstyle=\color{blue},       % keyword style
  language=MAKE,                    % the language of the code
  morekeywords={*,...},            % if you want to add more keywords to the set
  numberstyle=\tiny\color{mygray}, % the style that is used for the line-numbers
  rulecolor=\color{black},         % if not set, the frame-color may be changed on line-breaks within not-black text (e.g. comments (green here))
  showspaces=false,                % show spaces everywhere adding particular underscores; it overrides 'showstringspaces'
  showstringspaces=false,          % underline spaces within strings only
  showtabs=false,                  % show tabs within strings adding particular underscores
  stepnumber=2,                    % the step between two line-numbers. If it's 1, each line will be numbered
  stringstyle=\color{mymauve},     % string literal style
  tabsize=2,	                   % sets default tabsize to 2 spaces
  title=\lstname                   % show the filename of files included with \lstinputlisting; also try caption instead of title
}

\title{Computação Gráfica (3º ano de Curso)\\
       \textbf{Fase 4}\\ Relatório de Desenvolvimento
       } %Titulo do documento
%\title{Um Exemplo de Artigo em \LaTeX}
\author{Diogo Fernandes\\ (A87968) \and Luís Guimarães\\ (A87947)
         \and Ivo Lima\\ (A90214)
       } %autores do documento
\date{\today} %data

\begin{document}
	\begin{minipage}{0.9\linewidth}
        \centering
		\includegraphics[width=0.4\textwidth]{um.jpeg}\par\vspace{1cm}
		{\scshape\LARGE Universidade do Minho} \par
		\vspace{0.6cm}
		{\scshape\Large Licenciatura em Ciências da Computação} \par
		\maketitle
	\end{minipage}


\tableofcontents % insere Indice

\chapter{Introdução}
Uma vez que esta é a última fase foi pedido uma série de novas implementações. O \emph{generator} passou a gerar, para além das coordenadas dos pontos, as normais e as as coordenadas de textura. 

Para conseguirmos tirar partido dessa nova informação incluída nos modelos 3D, tivemos de ativar as luzes e as texturas no OpenGL e definir as propriedades de cada modelo, para assim utilizarmos este novo tipo de coloração.

Posto isto, estabelecemos um conjunto de tarefas que incidiram tanto sobre o \emph{generator} como a \emph{engine} para que esta possa suportar os requisitos indicados no enunciado e outros definidos por nós:

{\bfseries Atualizações no \emph{generator}:}
\begin{enumerate} 
\item Gerar as normais e as coordenadas de textura dos modelos atualizados;
\end{enumerate}

{\bfseries Atualizações na \emph{engine}:}
\begin{enumerate} 
\item Atualizar o parser de modo a dar suporte aos novos tipos de configuração xml e atualizar/criar estruturas de dados para guardar as propriedades das luzes, materiais e texturas;
\item Introduzir uma câmara fps, assim como outras features para facilitar o debug/visualização do ambiente;
\end{enumerate}
\chapter{Atualização do \emph{generator}}

Com a introdução desta fase, percebemos que para se dar a criação das normais e das coordenadas de textura apenas teriamos de fazer uma leve atualização na estratégia adotada até ao momento, ou seja, após a geração dos três pontos que definem um dos três vértices do triângulo serão introduzidas três coordenadas para as normais e outras duas que representarão os pontos da textura.

\section{Nova configuração de geração dos pontos}

Tendo isto posto, segue-se nas próximas secções uma breve descrição daquilo que foi alterado através da apresentação do código para gerar as normais e as coordenadas de textura.
\newpage
\subsection{\emph{Plane}}

Neste primeiro caso um dos mais simples foram acrescentadas as linhas apresentadas a seguir:

\begin{multicols}{2}
\begin{lstlisting}[style = code]
     ...
     << 0 << " " << 1 << " " << 0 << " "
     << 1 << " " << 0 << "\n";
     ...
     << 0 << " " << 1 << " " << 0 << " "
     << 1 << " " << 1 << "\n";
     ...
     << 0 << " " << 1 << " " << 0 << " "
     << 0 << " " << 1 << "\n";

     ...
     << 0 << " " << 1 << " " << 0 << " "
     << 1 << " " << 0 << "\n";
     ...
     << 0 << " " << 1 << " " << 0 << " "
     << 0 << " " << 1 << "\n";
     ...
     << 0 << " " << 1 << " " << 0 << " "
     << 0 << " " << 0 << "\n";

     ...
     << 0 << " " << -1 << " " << 0 << " "
     << 1 << " " << 1 << "\n";
     ...
     << 0 << " " << -1 << " " << 0 << " "
     << 1 << " " << 0 << "\n";
     ...
     << 0 << " " << -1 << " " << 0 << " "
     << 0 << " " << 1 << "\n";

     ...
     << 0 << " " << -1 << " " << 0 << " "
     << 0 << " " << 1 << "\n";
     ...
     << 0 << " " << -1 << " " << 0 << " "
     << 1 << " " << 0 << "\n";
     ...
     << 0 << " " << -1 << " " << 0 << " "
     << 0 << " " << 0 << "\n";
\end{lstlisting}
\end{multicols}
\begin{figure}[h]
    \centering
    \includegraphics[width=0.6\textwidth]{plane.png}
\end{figure}

\newpage
\subsection{\emph{Box}}

A \emph{Box} já começou a exigir um pouco mais de racíocínio e fez com que tivéssemos de acertar de alguns pontos e acabou ficando assim:

\begin{multicols}{3}
\begin{lstlisting}[style = code]
       ...
       << 0 << " " << 1 << " " << 0 << " "
       << 0.3 << " " << 0.5 << "\n";
       ...
       << 0 << " " << 1 << " " << 0 << " "
       << 0.3 << " " << 1 << "\n";
       ...
       << 0 << " " << 1 << " " << 0 << " "
       << 0 << " " << 1 << "\n";

       ...
       << 0 << " " << 1 << " " << 0 << " "
       << 0.3 << " " << 0.5 << "\n";
       ...
       << 0 << " " << 1 << " " << 0 << " "
       << 0 << " " << 1 << "\n";
       ...
       << 0 << " " << 1 << " " << 0 << " "
       << 0 << " " << 0.5 << "\n";

       ...
       << 0 << " " << -1 << " " << 0 << " "
       << 1 << " " << 0.5 << "\n";
       ...
       << 0 << " " << -1 << " " << 0 << " "
       << 1 << " " << 1 << "\n";
       ...
       << 0 << " " << -1 << " " << 0 << " "
       << 0.6 << " " << 0.5 << "\n";

       ...
       << 0 << " " << -1 << " " << 0 << " "
       << 0.6 << " " << 0.5 << "\n";
       ...
       << 0 << " " << -1 << " " << 0 << " "
       << 1 << " " << 1 << "\n";
       ...
       << 0 << " " << -1 << " " << 0 << " "
       << 0.6 << " " << 1 << "\n";

       ...
       << 1 << " " << 0 << " " << 0 << " "
       << 0.6 << " " << 0.5 << "\n";
       ...
       << 1 << " " << 0 << " " << 0 << " "
       << 0.6 << " " << 1 << "\n";
       ...
       << 1 << " " << 0 << " " << 0 << " "
       << 0.3 << " " << 1 << "\n";

       ...
       << 1 << " " << 0 << " " << 0 << " "
       << 0.6 << " " << 0.5 << "\n";
       ...
       << 1 << " " << 0 << " " << 0 << " "
       << 0.3 << " " << 1 << "\n";
       ...
       << 1 << " " << 0 << " " << 0 << " "
       << 0.3 << " " << 0.5 << "\n";

       ...
       << -1 << " " << 0 << " " << 0 << " "
       << 0 << " " << 0.5 << "\n";
       ...
       << -1 << " " << 0 << " " << 0 << " "
       << 0 << " " << 0 << "\n";
       ...
       << -1 << " " << 0 << " " << 0 << " "
       << 0.3 << " " << 0.5 << "\n";

       ...
       << -1 << " " << 0 << " " << 0 << " "
       << 0.3 << " " << 0.5 << "\n";
       ...
       << -1 << " " << 0 << " " << 0 << " "
       << 0 << " " << 0 << "\n";
       ...
       << -1 << " " << 0 << " " << 0 << " "
       << 0.3 << " " << 0 << "\n";

        ...
       << 0 << " " << 0 << " " << 1 << " "
       << 0.6 << " " << 0 << "\n";
       ...
       << 0 << " " << 0 << " " << 1 << " "
       << 0.6 << " " << 0.5 << "\n";
       ...
       << 0 << " " << 0 << " " << 1 << " "
       << 0.3 << " " << 0.5 << "\n";

       ...
       << 0 << " " << 0 << " " << 1 << " "
       << 0.6 << " " << 0 << "\n";
       ...
       << 0 << " " << 0 << " " << 1 << " "
       << 0.3 << " " << 0.5 << "\n";
       ...
       << 0 << " " << 0 << " " << 1 << " "
       << 0.3 << " " << 0 << "\n";

       ...
       << 0 << " " << 0 << " " << -1 << " "
       << 0.6 << " " << 0.5 << "\n";
       ...
       << 0 << " " << 0 << " " << -1 << " "
       << 0.6 << " " << 0 << "\n";
       ...
       << 0 << " " << 0 << " " << -1 << " "
       << 1 << " " << 0.5 << "\n";

       ...
       << 0 << " " << 0 << " " << -1 << " "
       << 1 << " " << 0.5 << "\n";
       ...
       << 0 << " " << 0 << " " << -1 << " "
       << 0.6 << " " << 0 << "\n";
       ...
       << 0 << " " << 0 << " " << -1 << " "
       << 1 << " " << 0 << "\n";
\end{lstlisting}
\end{multicols}
\begin{figure}[h]
    \centering
    \includegraphics[width=0.7\textwidth]{box.png}
\end{figure}

\subsection{\emph{Sphere}}

Nesta figura os cálculos ficaram da seguinte maneira:
\begin{multicols}{2}
\begin{lstlisting}[style = code]
       ...	
      << (radius * cos(next_b) * sin(next_a))/radius << " " << (radius * sin(next_b))/radius << " " << (radius * cos(next_b) * cos(next_a))/radius << " "
      << next_a / pi_mul_2 << " " << (next_b + pi_div_2)/ M_PI << "\n";
			
       ...
      << (radius * cos(next_b) * sin(next_a))/radius << " " <<  (radius * sin(next_b))/radius << " " << (radius * cos(next_b) * cos(next_a))/radius << " "
      << a / pi_mul_2 << " " << (next_b + pi_div_2)/ M_PI << "\n";

       ...
      << (radius * cos(next_b) * sin(next_a))/radius << " " << (radius * sin(next_b))/radius << " " << (radius * cos(next_b) * cos(next_a))/radius << " "
      << next_a / pi_mul_2 << " " << (b + pi_div_2) / M_PI << "\n";

       ...
      << (radius * cos(next_b) * sin(next_a))/radius << " " <<  (radius * sin(next_b))/radius << " " << (radius * cos(next_b) * cos(next_a))/radius << " "
      << next_a / pi_mul_2 << " " << (next_b + pi_div_2)/ M_PI << "\n";

       ...
      << (radius * cos(next_b) * sin(next_a))/radius << " " <<  (radius * sin(next_b))/radius << " " << (radius * cos(next_b) * cos(next_a))/radius << " "
      << a / pi_mul_2 << " " << (next_b + pi_div_2)/ M_PI << "\n";

       ...
      << (radius * cos(next_b) * sin(next_a))/radius << " " <<  (radius * sin(next_b))/radius << " " << (radius * cos(next_b) * cos(next_a))/radius << " "
      << a / pi_mul_2 << " " << (b + pi_div_2) / M_PI << "\n";
\end{lstlisting}
\end{multicols}
\begin{figure}[h]
    \centering
    \includegraphics[width=0.40\textwidth]{sphere.png}
\end{figure}
\newpage
\subsection{\emph{Cone}}
Por fim foi adicionado o seguinte:
\begin{multicols}{2}
\begin{lstlisting}[style = code]
       ...
      << 0 << " " << -1 << " " << 0 << " "
      << 1/6 << " " << 1/6 << "\n";
       ...
      << 0 << " " << -1 << " " << 0 << " "
      << 1/6 * sin(next_a) << " " << 1/6 * cos(next_a) << "\n";
       ...
      << 0 << " " << -1 << " " << 0 << " "
      << 1/6 * sin(a) << " " << 1/6 * cos(a) << "\n";
      ...

       ...
      << cos(atan(radius/height))*cos(next_a) << " " << sin(atan(radius/height)) << " " << cos(atan(radius/height))*sin(next_a) << " "
      << sin(next_a) * ((height - next_h) / height) * cos(next_a)  << " " << next_h/height << "\n";
       ...
      << cos(atan(radius/height))*cos(a) << " " << sin(atan(radius/height)) << " " << cos(atan(radius/height))*sin(a) << " "
      << sin(a) * ((height - next_h) / height) * cos(a)  << " " << next_h/height << "\n";
       ...
      << cos(atan(radius/height))*cos(next_a) << " " << sin(atan(radius/height)) << " " << cos(atan(radius/height))*sin(next_a) << " "
      << sin(next_a) * ((height - next_h) / height) * cos(next_a)  << " " << next_h/height << "\n";

       ...
      << cos(atan(radius/height))*cos(a) << " " << sin(atan(radius/height)) << " " << cos(atan(radius/height))*sin(a) << " "
      << sin(a) * ((height - next_h) / height) * cos(a)  << " " << next_h/height << "\n";
       ...
      << cos(atan(radius/height))*cos(a) << " " << sin(atan(radius/height)) << " " << cos(atan(radius/height))*sin(a) << " "
      << sin(a) * ((height - next_h) / height) * cos(a)  << " " << next_h/height << "\n";
       ...
      << cos(atan(radius/height))*cos(next_a) << " " << sin(atan(radius/height)) << " " << cos(atan(radius/height))*sin(next_a) << " "
      << sin(next_a) * ((height - next_h) / height) * cos(next_a)  << " " << next_h/height << "\n";  
\end{lstlisting}
\end{multicols}

\chapter{Atualização da \emph{engine}}

\section{Nova configuração do xml}
O ficheiro de configuração do \emph{xml} que temos vindo a utilizar deu suporte às luzes, que definirão a nossa \emph{scene} que podem ser definidas de 3 tipos distintos: \textbf{\emph{spot}}, \textbf{ \emph{point}} e \textbf{\emph{directional}}.

A definição das luzes deverá aparecer da seguinte forma: 

As \textbf{\emph{spotlight}} simulam uma luz parecida aquela vinda de uma lanterna, sendo que para obtermos tal funcionalidade devemos especificar um ponto, uma direção, a abertura da luz (cutoff) e um expoente para definir a sua intensidade.

\begin{lstlisting}[style = xml]
<lights>
    <light type="SPOT" posX="..." posY="..." posZ="..."
                       dirX="..." dirY="..." dirZ="..."
                       cutoff="..." exponent="..." />
<lights>
\end{lstlisting}

No caso das luzes de tipo \textbf{ \emph{point}} que são definidas num ponto e emitem em todas as direções. Terá a seguinte definição:

\begin{lstlisting}[style = xml]
<lights>
    <light type="POINT" posX="..." posY="..." posZ="..." />
</lights>
\end{lstlisting}

Já as \textbf{\emph{directional}} \emph{lights} não estão definidas num ponto pois seguem uma dada direção.
\begin{lstlisting}[style = xml]
<lights>
   <light type="DIRECTIONAL" dirX="..." dirY="..." dirZ="..." />
</lights>
\end{lstlisting}

\iffalse 
Para todos os tipos de luzes enunciados a cima podem especificar a intensidade RGB das luzes no que toca à componente difusa, ambiente e especular, assim:
\begin{lstlisting}[style = xml]
<light ... ambiR="..." ambiG="..." ambiB="..."
           specR="..." specG="..." specB="..."
     diffR="..." diffG="..." diffB="..." />
\end{lstlisting}
\fi

Os modelos passam a ser desenhados com cor, sendo possível especificar cada tipo de cor no modelo RGB:
\begin{lstlisting}[style = xml]
< model file = "..." ambiR="..." ambiG="..." ambiB="..."
		                 emisR="..." emisG="..." emisB="..."
                     specR="..." specG="..." specB="..."
                     diffR="..." diffG="..." diffB="..." />
\end{lstlisting}

Podem também ser desenhados modelos com uma determinada textura:
\begin{lstlisting}[style = xml]
< model file = "..." texture="..." />
\end{lstlisting}

Ou com cor e textura:
\begin{lstlisting}[style = xml]
< model file = "..." ambiR="..." ambiG="..." ambiB="..."
		                 emisR="..." emisG="..." emisB="..."
                     specR="..." specG="..." specB="..."
                     diffR="..." diffG="..." diffB="..."
                     texture="..." />
\end{lstlisting}

\section{Implementação}
Para que a iluminação e a texturização sejam possíveis, é necessário adicionar as seguintes linhas de código à função main:

\begin{lstlisting}[style = code]
glEnableClientState(GL_VERTEX_ARRAY);
glEnableClientState(GL_NORMAL_ARRAY);
glEnableClientState(GL_TEXTURE_COORD_ARRAY);

glEnable(GL_LIGHTING);
glEnable(GL_TEXTURE_2D);
\end{lstlisting}

Foram criados dois \emph{vectors}, um para armazenar as normais e outro para armazenar as coordenadas de textura.

Estes vectors são preenchidos na função {$readXML\_aux$} ao percorrer os ficheiros \emph{xxx.3d} associados aos modelos, que contêm, para cada ponto, as suas coordenadas, as normais e as suas coordenadas de textura.

Foram também criados mais dois buffers, um que armazena as normais e outro que armazena as coordenadas de textura.
\newpage
Ficamos então com três buffers na totalidade, que são gerados da seguinte forma:
\begin{lstlisting}[style = code]
glGenBuffers(3, buffers);
glBindBuffer(GL_ARRAY_BUFFER, buffers[0]);
glBufferData(
	GL_ARRAY_BUFFER, sizeof(float) * v.size(), v.data(), GL_STATIC_DRAW);

glBindBuffer(GL_ARRAY_BUFFER, buffers[1]);
glBufferData(
	GL_ARRAY_BUFFER, sizeof(float) * n.size(), n.data(), GL_STATIC_DRAW);

glBindBuffer(GL_ARRAY_BUFFER, buffers[2]);
glBufferData(
	GL_ARRAY_BUFFER, sizeof(float) * t.size(), t.data(), GL_STATIC_DRAW);
\end{lstlisting}

Estes buffers são depois ativados na função \emph{renderScene} da seguinte forma:
\begin{lstlisting}[style = code]
glBindBuffer(GL_ARRAY_BUFFER, buffers[0]);
glVertexPointer(3, GL_FLOAT, 0, 0);

glBindBuffer(GL_ARRAY_BUFFER, buffers[1]);
glNormalPointer(GL_FLOAT, 0, 0);

glBindBuffer(GL_ARRAY_BUFFER, buffers[2]);
glTexCoordPointer(2, GL_FLOAT, 0, 0);
\end{lstlisting}

\subsection{Luzes}
Foram definidas duas \emph{structs} que armazenam as informações sobre as luzes. LIGHT é usada para os tipos POINT e DIRECTIONAL, enquanto SPOTLIGHT é usada para o tipo SPOT. A distinção entre POINT e DIRECTIONAL está no último argumento de pos (1.0 se for POINT e 0.0 se for DIRECTIONAL).

\begin{lstlisting}[style = code]
struct LIGHT {
	 int n; //number
	 float pos[4];
};
struct SPOTLIGHT {
	 int n; //number
	 float pos[4];
	 float cutoff[1];
	 float exponent[1];
	 float spotDir[3]; };
\end{lstlisting}

Foram criados dois \emph{vectors} que armazenam as informações sobre as luzes.

\begin{lstlisting}[style = code]
vector<LIGHT> lights;
vector<SPOTLIGHT> spotlights;
\end{lstlisting}

Na função recursiva {$readXML\_aux$}, que trata de percorrer o ficheiro xmlconfig, foi adicionado um caso para "lights" que trata de armazenar a informação sobre as luzes na estrutura de dados correspondente.

As luzes são depois ativadas na \emph{renderScene} com recurso às funções glEnable e glLightfv do openGL.

\subsection{Cores}
Foram adicionadas as seguintes variáveis à struct FIGURE:
\begin{lstlisting}[style = code]
float dif[4];
float amb[4];
float emi[4];
float spe[4];
float shi;
\end{lstlisting}

Na função recursiva {$readXML\_aux$}, são adicionados à struct FIGURE correspondente os valores RBA das cores difusa, ambiente, emissiva, especular e brilho.
No caso de não ter sido definida alguma das cores, o seu valor será o default do openGL.

Estes valores são depois associados ao modelo na renderScene, com recurso à função glMaterialfv.

\subsection{Texturas}
Foi adicionada a seguinte variável à struct FIGURE:

\begin{lstlisting}[style = code]
GLuint texture;
\end{lstlisting}

Foi criada a função \emph{loadTexture} que recebe uma string como argumento e carrega e retorna a textura correpondente.

Na função {$readXML\_aux$}, no caso de ser especificado um ficheiro de textura, esta é carregada com recurso à função \emph{loadTexture} e guardada na struct FIGURE correspondente.

A textura será depois ativada na renderScene antes de desenhar cada modelo através de glBindTexture. No caso de não ter sido especificado um ficheiro de textura, glBindTexture receberá 0 como argumento o que corresponde a não desenhar textura.

\section{Outras implementações}

\subsection{CMakeLists}

Uma vez que o nosso projeto começou a crescer e a quantidade de \emph{features} também resolvemos gastar um pouco do nosso tempo a atualizar o \emph{CMakeList}. 

Para além da óbvia verificação e cópia da \emph{DevIL.lib} da pasta \emph{TOOLKITS\_FOLDER} resolvemos anular o trabalho de toda a vez que faziamos o \emph{Configure} e \emph{Generate} no programa \emph{CMake} ter de inserir manualmente na pasta \emph{build} os ficheiros com as figuras do tipo \emph{xxx.3d}, as texturas com a assinatura \emph{xxx.jpg} e ainda o \emph{xmlconf.xml}. Portanto por questões de organização esses ficheiro devem estar em pastas para que as suas diretivas possam ser passadas ao programa, a nova pasta \emph{FIGTEX\_FOLDER} deverá conter duas subpastas a \emph{textures} bem como a \emph{figs}, já o \emph{XML\_FOLDER} precisará conter uma subpasta \emph{config} com um ficheiro nomeado de \emph{xmlconf.xml}.
\begin{lstlisting}[style = cmake]

...

message(STATUS "Xml_DIR set to: " ${XML_FOLDER})
set(XML_FOLDER "" CACHE PATH "Path to XML folder")

message(STATUS "FigTex_DIR set to: " ${FIGTEX_FOLDER})
set(FIGTEX_FOLDER "" CACHE PATH "Path to FIGTEX folder")

...

if (EXISTS "${XML_FOLDER}")
		file(COPY ${XML_FOLDER}/config/xmlconf.xml DESTINATION ${CMAKE_BINARY_DIR})
endif(EXISTS "${XML_FOLDER}")

if (EXISTS "${FIGTEX_FOLDER}")
	    file(COPY ${FIGTEX_FOLDER}/textures/ DESTINATION ${CMAKE_BINARY_DIR})
	    file(COPY ${FIGTEX_FOLDER}/figs/ DESTINATION ${CMAKE_BINARY_DIR})
endif(EXISTS "${FIGTEX_FOLDER}")

\end{lstlisting}

O código escrito pede as diretivas das pastas para que possa copiar todos os ficheiros das suas subpastas e os cole na \emph{CMAKE\_BINARY\_DIR} que é a pasta \emph{build} gerada pelo \emph{CMake}.
\newpage
\subsection{Câmara fps}

Embora este não seja um dos aspetos mais importantes do trabalho resolvemos dedicar uns parágrafos à nossa implementação de uma câmara \emph{fps}.

O raciocínio passa por considerar que a câmara possui uma posição \emph{P} e um ponto para onde está a olhar e através desse conhecimento, criar o vetor da direção para onde a mesma está a olhar. Com esse vetor podemos reproduzir o movimento para a frente e para trás ( teclas $\uparrow$ e $\downarrow$ ) relativamente à direção do olhar.

Já o movimento lateral pode ser obtido calculando o vetor resultante do up com a direção do olhar multiplicando o resultado por -1 ou 1 caso seja para a esquerda ou direita ( teclas $\leftarrow$ e $\rightarrow$ ).

A mesma também consegue olhar à sua volta e rodar sobre o seu próprio eixo utilizando os ângulo alfa e beta para definir a posição do olhar com coordenadas esféricas ( movimento de \emph{drag} com o rato ).

As variáveis de controlo da câmara bem como a sua aplicação encontra-se definida em na \emph{engine.cpp}.

\chapter{Resultado Final}

Conseguimos portanto nesta última fase juntar tudo o que aprendemos o que culminou na criação deste magnífico Sistema Solar.

\begin{figure}[h]
    \centering
    \includegraphics[width=1.0\textwidth]{sistema_solar.png}
\end{figure}

\chapter{Conclusão}

Nesta última fase, o desenvolvimento faseado que o nosso grupo tem vindo a estabelecer termina, resultando num software com uma flexível configuração de novas \emph{features} em xml tirando partido do processamento que a API do OpenGL nos proporciona. Com isto queremos dizer que embora o resultado final seja muito importante, consideramos um bem maior o conseguir e conhecimento adquirido em cada uma das fases. O que deu uma oportunidade de aplicar os conceitos teóricos e comprovar o seu funcionamento através da formalização de código e algoritmos de geração de modelos, a implementação do movimento dos mesmos, as luzes, texturas,...

Em forma de resumo podemos dizer que esta última fase revelou-se a mais enriquecedora do projeto, pois atribuiu lhe um aspeto visual próprio ao nosso Sistema Solar através da implementação de texturas próprias para cada planeta e a implantação/definição do Sol como a única fonte de luz emissiva de todo o sistema. Concluimos então que todos os objetivos para este projeto foram cumpridos na totalidade, no entanto, a nossa \emph{engine} ainda poderia continuar a sofre modificações e aprimorações para possibilitar a introdução de cada vez mais conceitos/funcionalidades novas.

\end{document}
