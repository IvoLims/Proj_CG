
\documentclass[11pt,a4paper]{report}

\usepackage[portuges]{babel}
\usepackage[utf8]{inputenc} % define o encoding usado texto fonte (input)--usual "utf8" ou "latin1
\usepackage{graphicx} %permite incluir graficos, tabelas, figuras
\usepackage{subcaption}
\usepackage{listings}
\usepackage{color}
\usepackage{multicol}
\usepackage{indentfirst}

\definecolor{myblue}{rgb}{0.2,0.2,0.8}
\definecolor{mygray}{rgb}{0.5,0.5,0.5}
\definecolor{mymauve}{rgb}{0.58,0,0.82}

\lstdefinestyle{code}{ 
  backgroundcolor=\color{white},   % choose the background color; you must add \usepackage{color} or \usepackage{xcolor}; should come as last argument
  basicstyle=\footnotesize,        % the size of the fonts that are used for the code
  breakatwhitespace=false,         % sets if automatic breaks should only happen at whitespace
  breaklines=true,                 % sets automatic line breaking
  captionpos=b,                    % sets the caption-position to bottom
  commentstyle=\color{white},      % comment style
  deletekeywords={...},            % if you want to delete keywords from the given language
  escapeinside={\%*}{*)},          % if you want to add LaTeX within your code
  extendedchars=true,              % lets you use non-ASCII characters; for 8-bits encodings only, does not work with UTF-8
  firstnumber=1000,                % start line enumeration with line 1
  keepspaces=true,                 % keeps spaces in text, useful for keeping indentation of code (possibly needs columns=flexible)
  keywordstyle=\color{blue},       % keyword style
  language=C++,                    % the language of the code
  morekeywords={*,...},            % if you want to add more keywords to the set
  numberstyle=\tiny\color{mygray}, % the style that is used for the line-numbers
  rulecolor=\color{black},         % if not set, the frame-color may be changed on line-breaks within not-black text (e.g. comments (green here))
  showspaces=false,                % show spaces everywhere adding particular underscores; it overrides 'showstringspaces'
  showstringspaces=false,          % underline spaces within strings only
  showtabs=false,                  % show tabs within strings adding particular underscores
  stepnumber=2,                    % the step between two line-numbers. If it's 1, each line will be numbered
  stringstyle=\color{mymauve},     % string literal style
  tabsize=2,	                   % sets default tabsize to 2 spaces
  title=\lstname                   % show the filename of files included with \lstinputlisting; also try caption instead of title
}

\lstdefinestyle{xml}{ 
  backgroundcolor=\color{white},   % choose the background color; you must add \usepackage{color} or \usepackage{xcolor}; should come as last argument
  basicstyle=\footnotesize,        % the size of the fonts that are used for the code
  breakatwhitespace=false,         % sets if automatic breaks should only happen at whitespace
  breaklines=true,                 % sets automatic line breaking
  captionpos=b,                    % sets the caption-position to bottom
  commentstyle=\color{white},    % comment style
  deletekeywords={...},            % if you want to delete keywords from the given language
  escapeinside={\%*}{*)},          % if you want to add LaTeX within your code
  extendedchars=true,              % lets you use non-ASCII characters; for 8-bits encodings only, does not work with UTF-8
  firstnumber=1000,                % start line enumeration with line 1
  keepspaces=true,                 % keeps spaces in text, useful for keeping indentation of code (possibly needs columns=flexible)
  keywordstyle=\color{blue},       % keyword style
  language=XML,                 % the language of the code
  morekeywords={*,...},            % if you want to add more keywords to the set
  numberstyle=\tiny\color{mygray}, % the style that is used for the line-numbers
  rulecolor=\color{black},         % if not set, the frame-color may be changed on line-breaks within not-black text (e.g. comments (green here))
  showspaces=false,                % show spaces everywhere adding particular underscores; it overrides 'showstringspaces'
  showstringspaces=false,          % underline spaces within strings only
  showtabs=false,                  % show tabs within strings adding particular underscores
  stepnumber=2,                    % the step between two line-numbers. If it's 1, each line will be numbered
  stringstyle=\color{mymauve},     % string literal style
  tabsize=2,	                   % sets default tabsize to 2 spaces
  title=\lstname                   % show the filename of files included with \lstinputlisting; also try caption instead of title
}

\title{Computação Gráfica (3º ano de Curso)\\
       \textbf{Fase 3}\\ Relatório de Desenvolvimento
       } %Titulo do documento
%\title{Um Exemplo de Artigo em \LaTeX}
\author{Diogo Fernandes\\ (A87968) \and Luís Guimarães\\ (A87947)
         \and Ivo Lima\\ (A90214)
       } %autores do documento
\date{\today} %data

\begin{document}
	\begin{minipage}{0.9\linewidth}
        \centering
		\includegraphics[width=0.4\textwidth]{um.jpeg}\par\vspace{1cm}
		{\scshape\LARGE Universidade do Minho} \par
		\vspace{0.6cm}
		{\scshape\Large Licenciatura em Ciências da Computação} \par
		\maketitle
	\end{minipage}


\tableofcontents % insere Indice

\chapter{Introdução}
Uma vez que esta é a última fase foi pedido uma serie de novas implementações, foi introduzido ao \emph{generator} as coordenadas de textura e as respetivas normais de cada vértice das primitivas feitas  na primeira fase. 

Para conseguirmos tirar partido dessa nova informação incluída nos modelos 3D, tivemos de ativar as luzes no OpenGL e definir as propriedades de cada modelo, para assim utilizarmos este novo tipo de coloração.

Tais exigências obrigaram à reformulação da criação dos modelos, no \emph{generator}, acrescentando a cada ficheiro gerado os índices dos respetivos vértices para que posteriormente, na \emph{engine}, os modelos passem a ser desenhados utilizando o buffering dos índices.

Posto isto, estabelecemos um conjunto de tarefas que incidiram tanto sobre o \emph{generator} como a \emph{engine} para que esta possa suportar os requisitos indicados no enunciado e outros definidos por nós:

{\bfseries Atualizações no \emph{generator}:}
\begin{enumerate} 
\item Alterar a geração de todos os modelos introduzindo os índices no ficheiro gerado;
\item Aplicar algoritmos para gerar as coordenadas de textura e as normais dos modelos atualizados;
\end{enumerate}

{\bfseries Atualizações na \emph{engine}:}
\begin{enumerate} 
\item Alterar o processamento dos vértices até agora utilizado, passando de VBOs (sem índices) para VBOs (com índices);
\item Atualizar o parser de modo a dar suporte aos novos tipos de configuração xml e criar estruturas em memória para guardar as propriedades das luzes e materiais,...;
\item Introduzir uma câmara fps, assim como outras features para facilitar o debug/visualização do ambiente;
\end{enumerate}
\chapter{Atualização do \emph{generator}}
\section{Breves pensamentos}
Com a introdução desta última fase, percebemos que a criação das coordenadas de textura e as normais para cada vértice, a memória e o posterior carregamento dos dados no \emph{engine} utilizando a estratégia adotada até agora, não seria uma ideia muito boa em termos de custo computacional para a nossa aplicação. Portanto resolvemos implementar uma otimização na criação dos modelos sem a repetição dos vértices.
\section{Nova configuração de geração de modelos}

Posto isto segue-se uma breve descrição daquilo que foi alterado tanto a nível dos algoritmos dos modelos como a geração das normais, coordenadas de textura, entre outros.

\chapter{Atualização da \emph{engine}}

\section{Nova configuração do xml}
O ficheiro de configuração do \emph{xml} que temos vindo a utilizar deu suporte às luzes, que definirão a nossa \emph{scene} que podem ser definidas de 3 tipos distintos: \textbf{\emph{spot}}, \textbf{ \emph{point}} e \textbf{\emph{directional}}.

A definição das luzes deverá aparecer da seguinte forma: 

As \textbf{\emph{spotlight}} simulam uma luz parecida aquela vinda de uma lanterna, sendo que para obtermos tal funcionalidade devemos especificar um ponto, uma direção, a abertura da luz (cutoff) e um expoente para definir a sua intensidade.

\begin{lstlisting}[style = xml]
<lights>
    <light type="SPOT" posX="..." posY="..." posZ="..."
                       dirX="..." dirY="..." dirZ="..."
                       cutoff="4" exponent="100" />
<lights>
\end{lstlisting}
No caso das luzes de tipo \textbf{ \emph{point}} que são definidas num ponto e emitem em todas as direções. Terá a seguinte definição:

\begin{lstlisting}[style = xml]
<lights>
    <light type="POINT" posX="..." posY="..." posZ="..." />
</lights>
\end{lstlisting}
Já as \textbf{\emph{directional}} \emph{lights} não estão definidas num ponto pois seguem uma dada direção.
\begin{lstlisting}[style = xml]
<lights>
   <light type="DIRECTIONAL" posX="..." posY="..." posZ="..." />
</lights>
\end{lstlisting}
Para todos os tipos de luzes enunciados a cima podem especificar a intensidade RGBA das luzes no que toca à componente difusa, ambiente e especular, assim:
\begin{lstlisting}[style = xml]
<light ... ambiR="..." ambiG="..." ambiB="..."
           specR="..." specG="..." specB="..."
           diffR="..." diffG="..." diffB="..." />
\end{lstlisting}

\chapter{Conclusão}


\end{document}